% \documentclass[danish, a4paper, twocolumn, oneside]{memoir}
% \usepackage{A1_preamble}
\documentclass[A2_main.tex]{subfiles}

\begin{document}
\section{Teori}
Bølgefænomenet dukker op mange steder i hverdagen, især for dem som spiller musik, hopper i sjippetorv eller læser fysik. I laboratoriet genereres stående bølger på en snor og det er disse, som vi her vil beskrive. Men først må vi nærme os grundbegreberne
\subsection{Mekaniske bølger}
Når en substans (refereret som \emph{mediet}) forstyrres fra dets hvilestadie, vil hver enkel af mediets partikler undergå en forskydning omkring sit ligevægtspunkt, hvorfor der vil ske en energitransport i mediet. Dette er hvad i dagligtale kaldes en mekanisk bølge\footnote{Også andre bølger eksisterer, bølger som ikke behøver et medie at propagerer i. Her er lys et eksempel.}. Hvorvidt partiklerne forskydes langs bølgens udbredelsesretning eller på tværs, er hvad sepererer bølger i kategorierne \emph{transversale} og \emph{longitudinale} bølger.

I forsøget vil vi ikke betragte longitudinale bølger langs snoren, så derfor vil der ikke skrives mere om disse.

Fra skolegården kender mange, at en enkelt puls hurtigt døer ud langs sjippetorvet. Hvad er mere interessant er, hvis man med en drivkraft tvinger torvet til at undergå en periodisk bevægelse.

\subsection{Periodisk Bevægelse}
I forsøget betragtes en simpel harmonisk oscillator (SHM). Disse er særdeles simple at analysere og kaldes også sinusoidale bølger. Dette er fundamentet for alle bølger, som alle kan brydes op som en sum af sinusoidale bølger, jævnfør superposition.

Bølgen er en symmetrisk sekvens af bølge-- dale og toppe. Symmetrien opstår ved en bølgelængde, $\lambda$ (afstand fra vilkårligt punkt til selvsamme punkt i næste sekvens). Tiden det tager bølgen at gentage sig er perioden, $T=f^{-1}$, hvor $f$ er en nogenlunde pendant, svarende til antallet af bølger pr. tid (kaldet frekvensen). Da sinusbølgen kan relateres til en cirkulær bevægelse, introduceres også en vinkelfrekvens, $\omega = 2\pi f$ samt et mål for partiklernes maksimale udsving, nemlig amplituden, $A$.

Bølgen propagerer med hastigheden, $v = \lambda f$, og da forsøget omhandler en snor, da vil bølgen være bundet til en én--dimensionel bevægelse.

\subsection{Bølge på en streng}
I forsøget vil strengen være udstrukken, og ser man bort fra tyngdekraftens virkning (en fin antagelse), vil hvilestadiet af strengen være en ret linie. Indføres en $x$--akse langs snoren. Bølger på strengen er transversale, hvorfor partiklers forskydelse er langs $y$--aksen.Forskydelsen afhænger af position på strengen, men er ligeledes tidsafhængig, da bølgen propagerer med hastigheden $v\neq 0$. Bølgefunktionen er givet ved
\begin{equation}
    y(x, t) = A\cos(kx -\omega t), \quad k = \frac{2\pi}{\lambda}.
    \label{eq: wave}
\end{equation}
hvor $k$ indføres som bølgetallet.





\subsection{Stående bølger}




Frekvensen af en stående bølge, er givet ved
\begin{equation}
 f_g = \frac{1}{2L}\sqrt{\frac{F}{\mu}}
 \label{eq: swave}
\end{equation}
Hvor $F$ er snor spændingen, $\mu$ er masse pr. længdeenhed og $L$ er længden på snoren.

Dette er lige en lille test, til at se om det hele virker!
endnu en test
\end{document}
