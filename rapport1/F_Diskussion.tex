%\documentclass[danish, a4paper, twocolumn, oneside]{memoir}
%\usepackage{A1_preamble}
\documentclass[A2_main.tex]{subfiles}

\begin{document}
\section{Diskussion}
I både \cref{fig:frekvensSNor} og \cref{fig:frekvensMu} ses et datapunkt der afviger markant fra de teoretiske værdier. Det er svært at sige hvor fejlen kommer fra, men da errorbars viser, at den teoretiske linje ligger langt væk fra usikkerheden, må det skyldes en fejl i aflæsningen. Det kan også skyldes måske at vi kom til at dæmpe snoren ved at røre ved den under en afmåling.
\\ Ud over det, bidrog opsættelsen til nogle usikkerheder. Bølge-generatoren flyttede vi engang i mellem - dette bidrog til ændringer i dæmpningen og grundfrekvensen.
Ydermere kan det nævnes at vi antog endepunkterne til at være faste punkter, men i realiteten var der bølgebevægelse ud over det faste punkt, så vi egentlig har mistet energi her, men det har nok været en meget lille fejlkilde. Det har selvfølgelig også været en faktor at metoden hvorpå vi bestemte grundfrekvensen var rimelig tvivlsom idet at det var på øjemål. 
Sidst vil vi lige nævne at der altså var en del forvirring omkring nogle af snorenes labels. De var svære at læse og vi var i tvivl om flere, hvilket måske kan forklare nogle af vores mærkelige værdier, specielt den hvor vi varierede $\mu$, da vi så ville få et datapunkt hvor $\mu$ værdien ikke stemte overens med den grundfrekvens som vi målte.
\end{document}
