\documentclass[main]{subfiles}

\begin{document}
\section{Introduktion}
Denne rapport er skrevet i foråret d. \today, som et afsluttende eksamensprojekt i kurset \emph{Bølger og Optik}. Det undersøger en akusto--optisk modulator (AOM) og har til formål at bestemme hvilken krystal der bruges i den anvendte AOM. Til dette deles forsøget op i hovedsagligt to dele. Første del redegører for diffraktionsvinklen udfra målte afstande mellem nulteorden-- og førsteordenspletten, samt afstand fra AOM og skærm. Ud fra målingerne vil lydens hastighed i krystallen bestemmes.

Dernæst undersøges power output af lyd samt dens effekt på intensiteten af lysstrålerne ved første og anden plet.

I anden del vil samme lydhastighed bestemmes, men denne gang ved at bruge en switch. Der vil være en responstid fra at switchen tænder/slukker lyden, hvorfor der også er en tid før effekten ses. Denne effekt er afhængig af laserens bredde, hvorfor begrebet \emph{waist} introduceres..  

Til sidst vil usikkerhederne bag de to forsøgsdele diskuteres.

\end{document}
