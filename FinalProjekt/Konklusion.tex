\documentclass[main]{subfiles}

\begin{document}
\section{Konklusion}
I denne rapport undersøges den akusto optiske effekt i to delforsøg. Begge delforsøg har til formål at bestemme lydbølgens hastighed i den anvendte akusto-optiske-modulator (AOM). Først bruges en geometrisk tilgang, hvor afstanden mellem diffraktionsmønsterets nulte og første orden måles og derudfra bestemmes en seperationsvinkel, som nødvendigvis er afhængig af lydfrekvensen i krystallen (AOM'en). Med denne overvejelse bestemmes lydens hastighed i krystallet til at være $v_S=\SI{3377\pm89}{\meter\per\second}$.
Desuden blev det vist, at intensiteten af strålen ved første orden stiger tilsyneladende lineært.

I anden delforsøg betragtes den anvendte laser til at være en Gaussisk stråle, hvorfor den kan beskrives ved en enkelt parameter, strålens waist $w_0$. Sålænge AOM'en opretholdes vil der forkomme et diffraktionsmønster. Slukkes for frekvensgeneratoren, vil der  være en responstid som nødvendigvis er omvendt proportional med lydenshastighed og ligefrem proportional med strålens waist. Derfor, ved at måle intensitetens rise/fall time, vil man kun deducere lydens hastighed. Dette gjorde vi med mærkelige resultater som kan ses i \cref{BILAG}. Lydens hastighed fik vi til at være $v_S = \SI{586,3\pm138,5}{\meter\per\second}$. Denne afviger meget stort fra delforsøg $1$, hvilket kan forklares ved de mærkelige kurver som vist på \cref{BILAG}.

\end{document}
