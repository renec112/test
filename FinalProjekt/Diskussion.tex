\documentclass[main]{subfiles}

\begin{document}
\section{Diskussion}
\subsection{Optimering af dataopsamling}
Resultaterne bærer præg af, at det er første gang vi arbejdede med et rigtigt laser setup. Målingerne tog lang tid, som et resultat af at udstyret er meget fintfølsom og at der var mange sikkerhedsmæssige hensyn at tage ift. laseren. Forsøgsopstillingen var nogenlunde optimeret, men små ændringer viste stor virkninger på parametrene hvorfor det er oplagt at introducere systematiske fejl.

I modul et lader vi afstanden til skærmen være $l = \SI{29,8}{\centi\meter}$. En større afstand, ville resultere i at afstanden mellem laser-prikkene blev størrer hvorfor måleredskabets usikkerheder ville have mindre indflydelse, og præcisionen optimeres. Ud over det kunne der med fordel være opsat millimeter-papir på skærmen, for at kunne måle afstanden mere præcist.
\subsection{Usikkerheder og fejlkilder}
Forsøget er afhængig af flere paremetre og målinger. Hvori usikkeherne ikke kendes. Usikkerheden fra frekvensgeneratoren kendes ikke, men antages til at være negligibel lille. Andre målingers usikkerheder måtte skønnes - eksempelvis powermeteret der svingede omkring et punkt, deri skønnes usikkerheden til at være dette udsving. Endelig var der parametre, hvor usikkerheden ikke vides. $T_r$ var svær at vurdere, fordi usikkerheden både afhæng af hvor upræcist målepunkterne sættes og oscilloskopets system.
\\
I forsøget lavede vi direkte fejl. Vi var uopmærksomme på  parameteren $I_0$ fra ligning \cref{eq:Intensitet}. Og dermed haves for mange ukendte til at kunne plotte over det ønskede. Dernæst, da vi skulle måle waist ved hjælp at føre kniven hele vejen igennem laserstrålen (\cref{fig:graf3}), placerede vi ikke kniven i hvad der formodes til 
\\



\end{document}
