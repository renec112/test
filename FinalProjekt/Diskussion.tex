\documentclass[main]{subfiles}

\begin{document}
\section{Diskussion}
\subsection{Optimering af dataopsamling}
Resultaterne bærer præg af, at det er første gang vi arbejdede med et rigtigt laser setup. Målingerne tog lang tid, som et resultat af at udstyret er meget fintfølsom og at der var mange sikkerhedsmæssige hensyn at tage ift. laseren. Forsøgsopstillingen var nogenlunde optimeret, men små ændringer viste stor virkninger på parametrene hvorfor det er oplagt at introducere systematiske fejl.

I modul et lader vi afstanden til skærmen være $l = \SI{29,8}{\centi\meter}$. En større afstand, ville resultere i at afstanden mellem laser-prikkene blev størrer hvorfor måleredskabets usikkerheder ville have mindre indflydelse, og præcisionen optimeres. Ud over det kunne der med fordel være opsat millimeter-papir på skærmen, for at kunne måle afstanden mere præcist.
\subsection{Usikkerheder og fejlkilder}



\end{document}
