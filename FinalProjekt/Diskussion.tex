\documentclass[main]{subfiles}

\begin{document}
\section{Diskussion}
\subsection{Optimering af dataopsamling}
Resultaterne bærer præg af, at det er første gang vi arbejdede med et rigtig laser setup. Det betød, at flere ting ikke fungerede optimalt. Målingerne tog lang tid, som et resultat af at udsyret er meget fintfølsom, samtidig med, at der tog hensyn til laser-sikkerhed. Mange af parametrene var ikke optimeret til at give de bedste resultater. I model optimeres 1 orden med et powermeter til nogenludne høje værdier, men instrøktoren Andreas fik altid selv højere effektivitet.
\\ I forsøget ved modul et, kunne forsøget optimeres ydeligere. Der anvendes en afstand til skærmen på $l = 29,8 cm$. En større afstand, ville resultere i, at astanden mellem laser-prikkene blev størrer, og hermed ville præcisioen optimeres. Ud over det, kunne der med fordel være opsat milimeter papir på skærmen, for at kunne måle afstanden mere præcist.
\subsection{Usikkerheder og fejlkilder}



\end{document}
