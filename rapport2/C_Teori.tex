\section{Teori}
Optik er læren om lyset og dens opførsel. En underkategori af optik er geometrisk optik, som bruger geometri og trigonometri som primære redskaber. I dette forsøg vil der tages udgangspunkt i geometrisk optik, og hertil skal vigtige begreber som \emph{billeder}, \emph{linser} og \emph{brændvidder} defineres. Det vil blive gjort i de følgende afsnit.

\subsection{Objekter} 
Et objekt defineres som værende en genstand hvorfra lys kan radiere. De kan være selvluminøse eller reflekterende. Ligeledes skelnes der mellem punktobjekter, som ingen fysiske dimensioner har, og udvidede objekter, som indebærer alle realle objekter der har udbredte dimensioner. Sidstnævnte kan betragtes som en klumpning af et $N$ antal punktobjekter, hvor $N$ er et stort tal.


\subsection{Billeder}
Hvis et objekt udstråler lys på en reflekterende overflade, vil der af Snells lov gælde, at lyset reflekteres ved en vinkel. Det reflekterede lys vil danne et billede, hvorfor objektet ser ud til at komme fra et nyt punkt. Dette princip genkendes fra dagligdagen i spejle.

Billeder kan også dannes ved refraktion. Når lyset fra objektet bevæger sig i et stof med indeks $n_1$ og rammer den refrakterende overflade med indeks $n_2$, vil lyset alt efter forholdet mellem de to indicer enten spredes (såfremt $n_1 > n_2$) eller samles (hvis $n_1 < n_2$). Dette er princippet bag linser, hhv.\ konvekse og konkave.


I forsøget vil objektet være udstrakt og af formen som en pil. Højden fra spids til hale kaldes for $y$ af objektet og $y'$ af billedet. Ratioen af billedet og objektets højde definerer forstørrelseskonstanten, $m$, som

\begin{equation}
    m = \frac{y'}{y}
    \label{eq:m}
\end{equation}

Hvis objektets og billedets pile har samme retning, vil billedet være \emph{erekt}, og modsat kaldes det for \emph{invers}. For at gøre billedet inverst, må højderne have forskellig fortegn, hvorfor $m$ er negativ..

En vigtig egenskab ved billeder er, at de kan være objekter for en anden overflade og dermed et andet billede. Dette udgører en grundpille for geometrisk optik, og er fundamentet for mikroskoper samt refraktions teleskoper. 


\subsection{Fokal-- punkt og længde}
Det viser sig at parallele stråler konvergerer efter reflektion til et punkt
kaldet fokalpunktet (også kaldt brændpunktet), som betegnes $F$. For en sphærisk flade vil $F$ være i en afstand $\frac{R}{2}$ fra $V$, hvor $V$ er fladens vertex, altså centrum af kurven. Afstanden kaldes for fokallængden, og $R$ betegner her krumningsradien.

Dette er et  essentielt princip i optik, som har mange egenskaber.
Enhver stråle som er parallel med den optiske akse, vil refelkteres gennem $F$. Dette argument kan vendes, da lyset vil følge samme vej, hvorfor at enhver indkommende stråle, som løber gennem $F$ også vil reflekteres parallel med den optiske akse. Dette er eksakt for parabolske spejle, og en fin antagelse for sphæriske spejle såfremt strålerne er paraksiale (vinklen $\alpha$ er lille).

Det gælder, at forholdet mellem brændpunktets afstand, og objektets afstand

\begin{equation}
    \frac{1}{f} = \frac{1}{s} + \frac{1}{s'}
    \label{eq:fokalvokal}
\end{equation}

\subsection{Teleskopet}
Teleskopet har til formål at vise objekter ved store afstande. Der er forskellige former for teleskoper, men da der kun tages udgangspunkt i det refrakterende teleskop i forsøgsdel $2$, vil også kun denne beskrives i dette afsnit.

Et objekt som antages at være langt nok væk så paraksial approksimationerne gælder, stråler ind på en linse som samler strålerne til et reduceret billede af objektet. Dette billede skal nu være objekt for den næste linse, som forstørrer objektet i et virtuelt billede. Det er dette billede som bør opfattes med øjet.

Forstørrelseskonstanten defineres som ratioen af den vinkel øjet uden teleskop ville se billedet, samt vinklen med teleskopet. Dette kan også skrives
\begin{equation}
    M = - \frac{f_1}{f_2}
    \label{eq:angular magnification}
\end{equation}
At $M$ er negativ svarer til at billedet er inverst, altså ikke erekt. Det ses af \cref{eq:angular magnification}, at et teleskop skal have en lang fokallængde, $f_1$, og en kort objekt brændvidde, for at opnå en god forstørrelse. Dog skal diameteren af teleskopet også justeres, så $D$ forstørres ved store fokallængder.

For at opnå et erekt billede, som man kender det fra en kikkert, skal lyset reflekteres flere gange langs den vej fra objekt til øje.



