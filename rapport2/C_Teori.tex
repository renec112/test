\section{Teori}
Optik er læren om lyset og dens opførsel. Uden lys vil livet som vi kender det i dag ikke eksistere, og lys er hvad gør os i stand til at se vores omgivelser. Indenfor fysikken er der især brug for nye begreber så som \emph{billeder}, \emph{linser} og \emph{brændvidder}.

\subsection{Billeder} 
Et objekt kan være alt hvorfra lys kan radiere. De kan være selv--luminøse eller reflekterende. Ligeledes skelnes der mellem punkt objekter, som ingen fysisk dimension har, og udbredte objekter, som alle reælle objekter har en udbredte. Sidstnævnte kan dog betragtes som en klumpning af $N$ punkt objekter, hvor $N$ er et stort tal.

Hvis et objekt udstråler lys på en plan, reflekterende overflade, vil der af Snells lov gælde, at lyset reflekteres ved en vinkel, og da fladen er plan vil normalen være fastsat og vinklen bliver lig indfaldsvinklen.  

Det reflekterede lys ser ud til at komme fra et nyt punkt kaldet \emph{billedet}.
Dette er princippet bag et spejl.

Billeder kan også dannes ved refraktion. Når lyset fra objektet rammer den refrakterende overflade vil lyset spredes (såfremt $n_1 > n_2$ eller samles (hvis $n_1 < n_2$).
