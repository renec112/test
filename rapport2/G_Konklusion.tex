\section{Konklusion}
I delforsøg et blev to linser undersøgt, som hver havde en fokallængde på hhv. $5--$ og $\SI{10}{\centi\meter}$. Til sidstnævnte linse blev der indsamlet flere datapunkter, hvilket også kan ses på spredningen. Grunden til at der er så få målinger til linsen med fokallængde $\SI{5}{\centi\meter}$, er at fejlen blev for stor til bestemmelse af billedets position.

I delforsøg to blev et teleskop samlet med udgangspunkt i teorien om geometrisk optik, som beskrevet i teoriafsnittet. Det lykkedes os, at forstørre og invertere et billede.
