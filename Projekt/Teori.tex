\documentclass[main]{subfiles}
\begin{document}
\section{Teori}
\subsection{Diffraktion  af lysbølge}
Når en bølge af lys eller lyd propagerer igennem snævre åbninger på størrelse med bølgelængden, vil bølgen spredes i mønstre bestemt ud fra bølgeegenskaber som destruktiv og konstruktiv interferens. Dette bølgefænomen kaldes diffraktion, og er grundlæggende for denne rapport. Ved


\subsection{Akusto-Optisk Modulator}
En akusto-optisk modulator, består af en Piezo Elektrish Transducer (PZT), en gennemsigtig krystal og en akustisk absorber. Fra PZT'en genereres mekaniske vibrationer som propagerer longitudinelt op gennem krystallen som en lydbølge. Derfor vil der i krystallen være en tidsafhængig tæthedsfunktion (fra trykbølgen), og som konsekvens et moduleret brydningsindeks givet ved,
\begin{equation}
    n(x,t) = n_0 + \Delta n \cos(k_s x - \omega t).
    \label{eq:brydningsindeks}
\end{equation}
Resultatet af en lydbølge i et transparent medie er denne tæthedsfunktion, hvor $n_0$ er den uforstyrret brydningsindekset, $\omega$ er vinkelfrekvensen, $k=\frac{2\pi}{\lambda}$ er bølgetallet for lydbølgen og $\Delta n$ er variationen af amplituden i det brydningsindekset, genereret af lydbølgen.

Det er denne rejsende variation i krystallets tæthed, som skal fungere som diffraktionsgitter for den indkommende laser. Sidstnævnte absorber har til formål at absorbere størstedelen af lydbølgen, så der ikke vil reflekteres energi tilbage i krystallen.

Den genererede brydningsindeks giver anledning til et diffraktionsgitter som rejser med lydens hastighed i krystallet som medie. Lys som propagerer igennem det transparente materiale vil diffraktere grundet denne genererede brydningsindeks. Derfor forventes et diffraktionsmønster såfremt man stiller en skærm i en vis afstand fra AOM'en.

Her gælder de konventionele formler for konstruktiv interferens ved vinkler 
\begin{equation}
    d\sin\theta_m = m\lambda,
    \label{eq:konstruktiv}
\end{equation}

\begin{equation}
    f_n = f_0 \pm nf_s, \qquad n=0, \pm1, \pm2, \cdots
    \label{eq:frequence}
\end{equation}
hvor $f_0$ er den upåvirkede frekvens, $f_s$ er lydens frekvens og $n$ er den valgte orden. 

\subsection{Regimer}
Når lys diffrakteres af en lydbølge med en enkelt frekvens, vil der være to diffraktionstyper. Raman-Nath og Bragg Diffraktion.

Raman-Nath diffraktion observeres ved lave lydfrekvenser (omkring $\SI{10}{\mega\hertz}$ og smal bredde for den akusto-optiske interaktioner (bredden af krystallen). Denne type sker for vilkårlige indfaldsvinkler.

    Bragg diffraktion vil modsat være ved høj lydfrekvens (omkring $\SI{100}{\mega\hertz}$). For Bragg diffraktion vil der primært være to diffraktions maksima, nulte og første orden.

    For at skelne de to regimer bruges betingelsesværdien $Q>>1$ og $Q<<1$ respektivt for hhv. Bragg og Raman-Nath. Her er $Q$ Klein-Cook parameteren givet ved
    \begin{equation}
    Q = \frac{2\pi\lambda L f^2}{nv^2}=\frac{2\pi\lambda L}{n{\lambda_s}^2}
        \label{eq:KleinCook}
    \end{equation}
I forsøget vil det primært bruges første orden $n=1$, hvorfor Bragg-regimet er ønsket, da der her er mindst effekt-tab i laserstrålen. Bragg-diffraktion præger dog forsøget med sine betingelser for en specifik vidde af frekvenser, og dermed også lydens hastighed.

\begin{equation}
    \theta_{sep} = 2 \theta_B = \frac{\lambda f_s}{v_s}.
    \label{eq:sep}
\end{equation}


\subsection{Effektivitet}
I forsøget vil forholdet mellem lydbølgernes amplitude og laserens intensitet undersøges. Der gælder at intensiteten for $n=1$ er
\begin{equation}
    I_1 = I_0 \sin^2\left( \sqrt{\eta} \right)
    \label{eq:Intensitet}
\end{equation}
hvor $\eta = \frac{\pi^2}{4}\frac{P}{P_0}$, og $P_0 = \frac{\lambda^2}{2 n_2}\frac{H}{L}$. Her er $P$ effekten af laserstrålen, og desuden introduceres $n_2$ nu som den akusto-optiske koefficient. Den sidste brøk er ren geometrisk, hvor $H$ er højden af lydbølgen, og $L$ er den førnævnte interaktions længde (krystallens længde). 

Når det plotts, forventes et peak ved en bestemt $P=P_0$. 

\subsection{Hastighed for switch af lys}
Så længe lydbølgen opretholdes af PZT'en, vil der være diffraktion. Vil man måle hastigheden for at tænde / slukke for lyset, kan man slukke for PZT'en. Lydens tøven har den virkning, at det vil tage noget tid for sidste lydbølge at passere target, og denne tid må være proportional med lydenshastighed samt targets størrelse.
\begin{equation}
    T_R \propto \frac{A}{v_s}
    \label{eq:risetime}
\end{equation}

Fra geometrisk optik vil lys fokuseres til et punkt ved at propagerere gennem en konkav linse. Dette punkt kaldes fokalpunktet.
Der vil nu bruges en anden model, som beskrive dette punkt som en minimal afstand, og ikke et samlepunkt. Den minimale afstand kaldes for strålens \emph{waist}.
\subsection{Gaussisk Stråle}
I forsøget bruges til en god approksimation en monokromatisk laserstråle. Den kan tilnærme sig en Gaussisk stråle, da dens tranversale magnetiske-- og elektriske felt bølges amplitude kan beskrives ved en Gauss funktion. Den fundamentale tranversale gaussiske tilstand ($TEM_00$) beskriver 

De elektriske og magnetiske feltbølgers amplitude profiler er bestemt af en parameter, den førhen nævnte \emph{waist}, $w_0$.

Gaussisk stråle











\end{document}
