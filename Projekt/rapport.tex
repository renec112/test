\documentclass[a4paper, oneside, twocolumn, 11pt]{memoir}

% Preamble
\usepackage{preamble}

\usepackage[backend=biber, sorting=none, defernumbers=true, style=numeric]{biblatex}
\addbibresource{bib.bib}
\defbibheading{secbib}[\bibname]{%
    \section*{#1}
\markboth{#1}{#1}}

% Bruges kun til display af kode
\usepackage{minted}
\setminted{style=pastie,breaklines=true}
\newenvironment{code}{\captionsetup{type=listing}}{}

% Skal indsættes efter ovenstående pakke Minted
\usepackage[danish = guillemets, autostyle = true]{csquotes}

\title{Akusto Optisk Diffraktionsgitter}
\author{Rasmus Klitgaard \thanks{rasmus.klitgaard@post.au.dk}, Rene Czepluch\thanks{rene.czepluch@post.au.dk} og Laurits N. Stokholm\thanks{laurits.stokholm@post.au.dk}}
\date{\today}

\begin{document}
% Title og Abstract
% Ignorerer twocolumn til abstract
\begin{minipage}{\textwidth}
    \twocolumn[
        \maketitle
        \begin{onecolabstract}
            <++>
        \end{onecolabstract}
    ]
\end{minipage}
\saythanks{}
\section{Indledning}
Forsøget er delt op i to dele. Første del redegører for diffraktionsvinklen,
målt udfra afstande mellem første plet og nulte ordens plet, samt afstand fra
gitter og plade.

Dernæst undersøges power output af lyd samt dens effekt på intensiteten af
lysstrålerne ved første og anden plet.

Diskuterer usikkerheder, vinkel


\section{Teori}
<++>
\section{Eksperimentel Opstilling}
<++>
\section{Databehandling}
<++>

\section{Konklusion}
<++>

<++>
%\printbibliography[heading=secbib]
\onecolumn
\chapter*{Bilag}
\begin{code}
    \caption{Den skrevne Pythonkode.%
    \label{kode}}
    \inputminted[%firstline=, lastline=,
        %frame=single, framesep=2mm, fontsize=\footnotesize
        linenos% Spanning over more than one page!
    ]{python}{lab.py}
\end{code}

\end{document}


